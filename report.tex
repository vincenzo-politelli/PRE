\documentclass{article}

% format
\usepackage[a4paper, total={6in, 9in}]{geometry}                            

% math symbols
\usepackage{amsmath}
\usepackage{amsthm}
\usepackage{amssymb}
\usepackage[shortlabels]{enumitem} 
\usepackage{mathtools}
\usepackage{bbm}
\usepackage{todonotes}

% figures
\usepackage{graphicx}
\usepackage{subcaption}

% theorems
\theoremstyle{definition}
\newtheorem{thm}{Theorem}[section]
\newtheorem{prop}[thm]{Proposition}
\newtheorem{lemma}[thm]{Lemma}
\newtheorem{rmk}[thm]{Remark}
\newtheorem{defn}[thm]{Definition}
\newtheorem{cor}[thm]{Corollary}
\newtheorem{exo}[thm]{Exercise}

% definition equal
\newcommand{\defeq}{\vcentcolon=}
\newcommand{\eqdef}{=\vcentcolon}

% bibliography
\usepackage[backend=biber,style=alphabetic, sorting=ynt]{biblatex}

% url highlight
\usepackage{hyperref}

% algos
\usepackage[linesnumbered,ruled,vlined]{algorithm2e}
\SetArgSty{textnormal}

% bullet point style
\renewcommand{\labelitemi}{\tiny$\blacksquare$}

% new commands
\DeclareMathOperator{\dom}{dom}
\DeclareMathOperator{\Hess}{\textbf{H}}
\DeclareMathOperator{\Diag}{Diag}
\DeclareMathOperator{\Tr}{Tr}
\DeclareMathOperator{\ind}{i}
\DeclareMathOperator{\sgn}{sign}

\addbibresource{../ref.bib}

\title{Linear recognition of almost interval graphs}
\author{Vincenzo Politelli \\ Supervisor: Prof. Chien-Chung Huang}
\date{}

\begin{document}

    \maketitle
    
    \section{Introduction and first definitions}
   
    In this report,
    we are going to provide
    a pedagogical presentation
    of the vertex deletion algorithm
    for interval graphs presented in \cite{main}.
    
    First of all, 
    we introduce the notation we will
    adopt in this text.
    Given a graph $G$ we
    are going to denote
    its set of vertices (resp. edges)
    by $V\left(G\right)$
    (resp. $E\left(G\right)$).
    We also denote $\left|V\left(G\right)\right|$ 
    by $\left|G\right|$ and
    $\left|E\left(G\right)\right|$
    by $\|G\|$.
    Given a vertex $v$ of
    $G$ we will denote its
    closed (resp. open)
    neighborhood in $G$ 
    by $N\left[v\right]$ 
    (resp. $N\left(v\right)$).
    We say that a vertex
    $v$ of $G$ is simplicial
    when $N\left[v\right]$ 
    induces a clique in $G$.
    We will denote the set
    of simplicial vertices of $G$ 
    by $S\left(G\right)$.

    We now introduce the 
    notion of interval graph.

    \begin{defn}
        We say that a graph $G$
        is interval if there 
        exists a family of intervals
        $\mathcal{I} = \left(I_{v}\right)_{v \in V\left(G\right)}$
         such that
         for all $u, v \in I_{v}$,
         $u \sim v$ if and only if
         $I_{u} \cap I_{v}$.
         The family $I$ is called
         an \emph{interval model}
         of $G$.
    \end{defn}

    Throughout this text,
    we will mostly assume
    without loss of generality
    that, when providing
    an interval model of a graph,
    all intervals are closed
    and bounded and all their
    non-empty intersections
    are not singletons.

    As proved in \cite{holes-caws},
    interval graphs
    are exactly
    the hole-free graphs
    (also known as chordal graphs)
    which do not have
    asteroidal triples.
    We remind that a hole is a
    chordless cycle and an
    asteroidal triple (from 
    now on ``at'')
    is a triple of vertices
    $\left\{x, y, z\right\}$ 
    such that for all
    couple of vertices in
    the said triple, 
    there exists a path
    connecting them
    that avoids the neighbours
    of the third one.
    The authors of \cite{holes-caws},
    went further in identifying
    the minimal chordal graphs
    (under the operation
    of taking iduced subgraphs)
    containing an at.
    Such graphs are shown in
    Figure \ref{caws} and are called
    minimal chordal asteroidal
    witnesses, caws for short.
    All in all, holes and caws
    are the minimal forbidden
    induced subgraphs for 
    interval graphs.
    We will denote the set of
    holes and caws by $\mathcal{F}_{I}$.

    \begin{figure}[h]
        \centering
        \includegraphics[width=\textwidth]{../caws.pdf}
        \caption{Minmimal chordal asteroidal witnesses (caws)}
        \label{caws}
    \end{figure}


    We now provide some
    terminology for caws.
    The vertices of caws
    which are colored in
    Figure \ref{caws} are called
    \emph{terminals},
    those which are
    colored in green
    are called
    \emph{shallow terminals}.
    When dealing
    with $\dagger_{d}$ or
    $\ddagger_{d}$,
    the set of vertices
    $\left\{l, b_1, c_1, c_2, s, b_d, r\right\}$
    (where $c_1$ and $c_2$ are
    identified in the case of $\dagger_{d}$)
    is called \emph{frame} of the caw,
    whereas the $b_1 - b_{d}$
    path is called the \emph{base}
    of the caw.

    Associated to the family 
    of interval graphs is the
    family of \emph{locally} interval
    graphs. That is, the family
    characterized by the set of forbidden
    induces subgraphs
    composed of holes
    of six or more vertices
    and caws which are
    not $\dagger_{d}$ or
    $\ddagger_{d}$ with
    $d \geq 3$.
    We will denote
    this new family of
    forbidden induced
    subgraphs by
    $\mathcal{F}_{LI}$.
    The holes which
    are contained
    in $\mathcal{F}_{LI}$ 
    are called \emph{short}
    holes, those which
    are not are called
    \emph{long} holes;
    the caws which
    are contained
    in $\mathcal{F}_{LI}$ 
    are called \emph{small} 
    caws and those
    which are not
    are called \emph{large}
    caws.

    One last family 
    of graphs that we
    will need is that
    of normal and Helly 
    circular-arc graphs.
    Similarly to interval
    graphs, circular-arcs
    graphs are those
    graphs which
    are represented
    by a \emph{circular arc 
    model} $\mathcal{A} = \left(A_{v}\right)_{v \in V\left(G\right)}$
    where $A_{v} = \left[l_{v}, r_{v}\right]$
    represents a closed arc
    with $l_{v}$ and $r_{v}$ respectively
    the counterclockwise and clockwise
    ends of the arc. We also
    assume $l_{v}, r_{v} > 0$ and
    so, $l_{v} > r_{v}$ in case
    the arc passes through the point $0$.
    Circular-arc graphs, however, 
    are far less understood than 
    interval graphs because of
    some pathological cases.
    This motivates the definition of
    normal and Helly circular-arc
    graphs.

    \begin{defn}
        A graph $G$ is
        a normal and Helly circular-arc
        graph if for all
        circular-arc models
        of $G$,
        no two arcs intersect 
        at both ends and
        no three arcs 
        pairwise intersect without
        sharing a common point.
    \end{defn}

    An important characterisation
    of normal and Helly circular-arc
    graphs is provided by \cite{nor-Helly-char}.
    We have that a graph $G$ is a normal
    and Helly circular-arc graph
    if and only if for all
    circular-arc models of $G$ 
    no three or less arcs cover the
    whole circle.

    \section{Vertex deletion problems that are FPT}
    
    The question we would like to 
    answer in this text is that of
    finding an optimal way of modifying
    a given graph into an interval graph.
    This problem is a special case 
    of a large and important family of problems
    that we are going to describe.

    \begin{defn}
        Let $\mathcal{C}$ be a class of graphs.
        \begin{itemize}
            \item If $\mathcal{C}$ has the property that, for any $G \in \mathcal{C}$,
                any induced subgraph of $G$ is also in $\mathcal{C}$,
                then we say that $\mathcal{C}$ is a hereditary property.
            \item If there exists a family $\mathcal{F}$ such that
                a graph $G \in \mathcal{C}$ if and only if $G' \not \in \mathcal{F}$
                for any induced subgraph $G'$ of $G$, then $\mathcal{C}$ 
                is said to have a forbidden set characterization.
                If, moreover, $\mathcal{F}$ is finite, $\mathcal{C}$ is 
                said to have a finite forbidden set characterization.
        \end{itemize}
    \end{defn}

    Notice that that of being an interval graph
    is a hereditary property
    and the class of interval
    graphs has a (non-finite) forbidden
    set characterization.
        
    Given a hereditary property $\mathcal{C}$,
    a natural problem that arises
    is the $\mathcal{C}$-\emph{graph modification problem}, 
    that is, the problem of computing the minimum
    number of vertices additions, edge additions or 
    edge deletions needed to transform a given graph $G$ into
    a graph in $\mathcal{C}$. 
    Notice that the fact that $\mathcal{C}$ is 
    a hereditary property
    makes the edge addition problem trivial.
    
    From now on, we are only going 
    to consider the vertex deletion problem. 
    This is because, if a graph $G$ can be made
    interval with $k_{+}$ edge additions
    and $k_{-}$ edge deletions,
    then it can be made so with $k_{+} + k_{-}$ 
    vertex deletions.
    Thus, knowing how to solve
    the vertex deletion problem 
    with $k$ deletions, we know
    how to solve the edge addition
    (resp. deletion) with $k$ 
    additions (resp. deletions).

    This kind of problems are hard in general, indeed, as proved in 
    \cite{NP-hard}, the vertex deletion problem is NP-hard
    for any non-trivial hereditary property.
    This motivates the research for FPT or approximation 
    schemes for the vertex-deletion problem. 
    As proved in \cite{FPT-algos}, the
    $\mathcal{C}$-vertex deletion problem
    for a class $\mathcal{C}$ which has
    a finite forbidden set
    characterization is indeed
    FPT in the maximum number of vertex deletions.
    In what follows, we are presenting a proof of this fact.

    If $\mathcal{C}$ has 
    a finite forbidden set characterization,
    let $\mathcal{F}$ be the set of forbidden induced subgraphs and
    let $\nu$ the maximum number of vertices of a graph in $\mathcal{F}$.
    Given a graph $G$ on $n$ vertices and $m$ edges,
    we can check in $\mathcal{O} \left(n^{\nu}\right)$
    whether $G$ belongs to the class $\mathcal{C}$.
    Indeed, for each graph $F$ in $\mathcal{F}$ 
    we can enumerate the functions from $V\left(F\right)$
    to $V\left(G\right)$ which are at most $n^{\nu}$
    and check in time $\mathcal{O} \left(\nu^{2}\right) = \mathcal{O}\left(1\right)$ whether
    any of them is an isomorphism.

    We get the following proposition.

    \begin{prop} \label{FPT-prop_1} (\cite{FPT-algos})
        For any hereditary property $\mathcal{C}$,
        if $\mathcal{C}$ is recognizable
        in time $T\left(n, m\right)$, with
        $n$, $m$ respectively the number of nodes and edges
        of the graph; then, for any $G \not \in \mathcal{C}$,
        we can find a minimal forbidded induced
        subgraph of $\mathcal{C}$ in G in time 
        $\mathcal{O} \left(n \cdot T\left(n, m\right)\right)$.  \end{prop}
    \begin{proof} 
        Suppose that there exists an algorithm $M$ that tests
        the membership of its argument $G$ in $\mathcal{C}$ in time
        $T\left(n, m\right)$.
        We first provide an algorithm and then prove its correcteness
        and time complexity.

        \begin{algorithm} \label{forbid subgraph}
            \caption{Minimal forbidden induced subgraph}
            \KwIn{An graph $G$ and a hereditary property $\mathcal{C}$.}
            \KwOut{An (inclusion-wise) minimal forbidden
            induced subgraph of $G$, otherwise the empty graph.}
           
            \SetKw{choose}{choose}

            $F \gets \emptyset$\;
            $V \gets V\left(G\right)$\;
            \While{$M\left(G\right)$ = ``False''}{
                \choose any vertex $v \in V$\;
                \If{$M\left(G - v\right)$}{
                    $F \gets F \cup \left\{v\right\}$\;
                } 
                \Else{
                    $G \gets G - v$\;
                }
            }
            \KwRet{$G\left[F\right]$}.
        \end{algorithm}
    The proof of the time complexity is clear.  
    The correctness follows from the fact that if
    $M\left(G\right) = \text{``False''}$ and $M\left(G - v\right) = \text{``True''}$,
    then $v$ is a node of an inclusion-wise minimal forbidden induced subgraph.
    \end{proof}
    
    From this we get that Algorithm \ref{vertex deletion FPT}
    is FPT and solves the vertex deletion problem. 
    
    \begin{algorithm} \label{vertex deletion FPT}
        \caption{Vertex deletion problem}
        \SetKw{branch}{branch}
        \SetKw{find}{find}

        \KwIn{A graph $G$ and a hereditary property $\mathcal{C}$ 
              with a finite forbidden set characterization;
              an integer $k$ representing the maximum number of
              vertex deletions.}
        \KwOut{``True'' if $G$ can be made into a graph in $\mathcal{C}$ 
               in at most $k$ vertex deletions, ``False'' otherwise.}
        $\texttt{count} \gets 0$\;
        
        \While{$\texttt{count} < k$}{
            \find a minimal induced forbidden subgraph $H$ in $G$
            with Algorithm \ref{forbid subgraph}\;
            \If{$H$ is found}{
                \branch on deleting a vertex from $H$\;
            }
            \Else{
                \KwRet{``True''}\;
            }
            $\texttt{count++}$\;
        }
        \KwRet{``False''}.
    \end{algorithm}

    We analise the algorithm.
    Given a graph $G$ and a hereditary property $\mathcal{C}$ with
    a finite forbidden set characterization, Proposition \ref{FPT-prop_1} and
    the previous observations tell us that the total complexity of the 
    algorithm, whose correctness is clear, is $\mathcal{O} \left(\nu^{k} \cdot n \cdot n^{\nu} \right)
    = \mathcal{O} \left(\nu^{k} \cdot n^{\nu + 1}\right)$.
    Which gives us the desired FPT algorithm.
    
    The class of interval graphs and many other important
    graph classes such as trees and chordal graphs 
    have the hereditary property but
    do not have a finite
    forbidden set characterization since they have arbitrarily big minimal
    obstructions such as cycles. Interval graphs also have
    very far less understood arbitrarily big minimal obstructions
    such as caws. Therefore the approach of Algorithm \ref{vertex deletion FPT}
    does not serve our purpose. Moreover, such algorithm is
    also very inefficient on special cases such as interval graph.

    One possible stategy to tackle this problem is the
    so-called \emph{two-phase approach}.
    That is, one first divides the class of forbidden
    induced subgraphs into two disjoint classes $\mathcal{F} = \mathcal{F}_{B} \cup \mathcal{F}_{U}$
    such that $\mathcal{F}_{B}$ contains some 
    selected minimal obstructions
    and $\mathcal{F}_{U}$ contains
    the rest of the graphs in $\mathcal{F}$.
    We are going to call the graphs in $\mathcal{F}_{B}$ 
    the \emph{small} obstructions and the graphs in $\mathcal{F}_{U}$ 
    the \emph{large} obstructions. 
    Phase 1 consists of disposing of small obstructions while,
    in phase 2, we get rid of large 
    ones, exploiting combinatorial properties of the graphs
    which lack small forbidden induced subgraphs.

    One example of this two-phase approach is
    that of the algorithm provided in \cite{aux1}.
    It is a linear time algorithm for detecting a hole
    or an at form a non-interval graph.
    It first calls the linear time hole detection algorithm
    described in \cite{aux2} (phase 1). 
    This either returns
    a hole, or reduces the problem to finding an at in a 
    chordal graph (phase 2).
    As noted in \cite{main}, we can 
    turn the algorithm from
    \cite{aux1} into a linear time 
    algorithm which returns either
    a hole or a caw. We will
    refer to this algorithm 
    as $\texttt{find\_forbidden}\left(\cdot\right)$.
    We will also be calling the
    hole-detecting algorithm
    in \cite{aux2}
    $\texttt{find\_hole}\left(\cdot\right)$.

    The problem of the two-phase approach is the fact that,
    in general, small obstructions are not likely to
    be found in linear time. 
    Thus, the natural choice
    of setting $\mathcal{F}_{B} \defeq \mathcal{F}_{LI}$
    and carry out the two-phase
    approach described above
    does not lead to optimal
    results.
    In order to achieve linear
    time in the interval vertex deletion problem
    we relax phase 1 to either disposing of
    an induced subgraph of $G$ in $\mathcal{F}_{LI}$
    or finding a certain clique decomposition
    of $G$ which allows us to remove
    the remaining obstructions.
    As an important note on what
    follows, we remark
    that whenever we
    find an induced subgraph of $G$ 
    in $\mathcal{F}_{LI}$ 
    during the execution
    of the vertex deletion
    algorithm, we immediately
    branch on deleting one vertex of it.

    We finally note that
    both Algorithm \ref{vertex deletion FPT}
    and the vertex deletion algorithm in \cite{main}
    pose themselves the constraint
    of modifying the input graph
    with at most $k$ modifications.
    What they do in practice, however,
    is to find the minimal number
    of modifications needed so that
    the input graph falls into the
    desired class if such number
    does not exceed $k$, or otherwise
    return ``False''.
    
    Thus, we assume 
    without loss of generality
    that the input graph
    $G$ does not contain
    any universal vertices.
    Indeed, if it does, 
    we can strip them off,
    because any minimal
    modification of $G$ 
    into an interval graph
    does not include
    universal vertices.
    To see why this is true,
    suppose that there is a 
    universal vertex $u$ into 
    a minimal set $S$ of 
    vertices to be deleted
    to make $G$ into an
    interval graph $G_{I}$.
    $G_{I}$ has an interval
    model $\mathcal{I} = \left(I_{v}\right)_{v \in V\left(G_{I}\right)}$.
    Adding the interval
    $\bigcup_{v \in V\left(G_{I}\right)} I_{v}$
    to $\mathcal{I}$ gives
    an interval model
    of $G \setminus \left(S \setminus \left\{u\right\}\right)$
    which contradicts
    the minimality of
    $S$.
    
    \section{Modules}

    One important notion behind
    the vertex deletion algorithm in
    \cite{main} is that of
    the module of a graph. 

    \begin{defn}
        Given a graph $G$, a set $M \subseteq V\left(G\right)$
        is a module if all $v \in M$ have
        the same neighborhood outside $M$.
        We call all singletons and $V\left(G\right)$
        \emph{trivial} modules.
        
        $G$ is said to be
        \emph{prime} if
        it only has trivial modules.

        A module $M$ is said to 
        be \emph{strong} if for all modules
        $M'$ that intersect $M$,
        either $M \subseteq M'$ or
        $M' \subseteq M$.

        Also, a module $M$ is \emph{maximal} if 
        the only module of $G$ properly
        containing it is $V\left(G\right)$.
    \end{defn}
    
    We provide some observations.
    One can check by hand
    that the $4$-hole is the
    only non-prime
    graph in $\mathcal{F}_{I}$.

    Notice that prime graphs with 
    at least one edge are always
    connected. Otherwise, any
    of it's connected components
    with at least two vertices
    would form a non-trivial module
    of such graph.
    
    Notice that two modules of a graph
    $G$ are either non-adjacent 
    or completely adjacent.

    We also immediately see that two modules
    which are both strong and maximal
    are disjoint. Moreover, since
    singletons are strong, every vertex
    $v$ appears in one maximal strong
    module. These observations give
    the following propositions.

    \begin{prop}
        The maximal strong modules of
        a graph $G$ define a partition
        of its vertices.
    \end{prop}

    Thus, maximal strong modules
    of $G$ define an equivalence
    relation on its vertices.
    Morever, since two modules
    are either non-adjacent
    or completely adjacent, 
    the maximal strong modules
    of a graph $G$ define a
    quotient graph $Q\left(G\right)$ 
    whose nodes are the maximal 
    strong modules of $G$
    which are connected in $Q\left(G\right)$ 
    if and only if they are
    completely adjacent in $G$.
    
    For our purposes, an important
    property of maximal strong modules,
    is that we can find them in linear
    time (see \cite{linear_modules}).
    Thus, they can be used for algorithmic
    purposes.
    We denote the linear-time
    subroutine computing
    $Q$ by $\texttt{compute\_quotient}\left(\cdot\right)$.

    We immediately see that
    if $G$ is not connected, its 
    maximal strong modules are
    its connected components. 
    Thus, $Q\left(G\right)$ is
    an edgeless graph.
    Likewise,
    if the complement of $G$,
    which we call $\overline{G}$, is
    not connected, we get
    that $Q\left(G\right)$ is 
    a complete graph. In general, 
    it holds that
    $G$ and $\overline{G}$ 
    have the same set of maximal
    strong modules
    and $Q\left(\overline{G}\right) = \overline{Q\left(G\right)}$. 

    If neither $G$
    nor $\overline{G}$ are
    connected, we have the following
    proposition.

    \begin{prop} \label{prime quotient}
        If $G$ and its complement are connected, 
        then the quotient graph  $Q$ whose vertex set is
        a set of maximum strong modules that partition $V\left(G\right)$ is prime.
    \end{prop}
    \begin{proof}
        Given a module $M$, let $\Gamma\left(M\right)$ denote
        the set of neighbors of the vertices of $M$ 
        which do not lie in $M$.
        Given a graph  $G$, let  $Q\left(G\right)$ denote
        its quotient graph  where the chosen partition of
        $V\left(G\right)$ is made of maximum strong modules.
        Notice that $G$ is connected if and only if $Q\left(G\right)$ is.
        Finally, let $\overline{G}$ be the complement of $G$.

        Now, suppose that $G$ and $\overline{G}$ are connected.
        Let $M = \left\{M_1, M_2, \ldots, M_{p}\right\}$
        be the maximum strong modules that form a partition of $V\left(G\right)$
        and let $Q\left(G\right)$ be the associated quotient graph.
        Suppose that $Q\left(G\right)$ contains a non-trivial module
        $\mu_1 \subseteq M$. Since the set $L_1 = \bigcup_{M_{i} \in \mu_{i}}$
        is a module in $G$ and $M_{i} \subseteq L_1$ for some $i$,
        we have that $L_1$ is not a strong module of G
        by the maximality of the modules in $M$.

        Therefore, there exists a module $L_1' \subseteq V\left(G\right)$
        such that $L_1' \cap L_1 \neq \emptyset$, 
        but none of the two sets is included into the other.
        Since the modules in $M$ are all strong,
        we immediately have that $L_1'$ is the disjoint union of some modules in $M$.
        Call the set of such modules $\mu_1'$.

        We can see that $\mu_1'$ is a module in $Q\left(G\right)$ and 
        $\mu_1 \cap \mu_1' \neq \emptyset$.
        Moreover, $\mu_1$ and $\mu_1'$ are not comparable with inclusion.

        Notice that $\mu_1 \cup \mu_1'$ must be a proper subset of $M$,
        because, otherwise, any vertex in the intersection of
        $\mu_1 \cap \mu_1'$ would be a disconnected component of 
        $Q\left(\overline{G}\right)$,  which cannot be since 
        $\overline{G}$ is connected.

        If $\Gamma\left(\mu_1\right) \subseteq \mu_1'$, then $\mu_1 \cup \mu_1'$ 
        would be a component of $Q\left(\overline{G}\right)$,
        but this would mean that $Q\left(\overline{G}\right)$
        has more than one connected component,
        since $\mu_1 \cup \mu_1'$ is a proper subset
        of $M$, as proved above. Therefore $\Gamma\left(\mu_1\right) \not \subseteq \mu_1'$.

        All in all, we proved that $\mu_1 \cup \mu_1' \defeq \mu_2$
        is a non-trivial module of $Q\left(G\right)$. 
        We can do the same on $\mu_2$ recursively until we obtain that
        $\mu_{l} \cup \Gamma\left(\mu_{l}\right)$ forms a partition of $M$ 
        with $\mu_{l}$ a non-trivial module of $Q\left(G\right)$, 
        and so $\Gamma\left(\mu_{l}\right) \neq \emptyset$.
        But this means that $\Gamma\left(\mu_{l}\right)$ forms
        a disconnected component of $Q\left(\overline{G}\right)$, which is a contradiction.
        Thus, $Q\left(G\right)$ does not have any non-trivial modules.
    \end{proof}
    
    We automatically have the following corollary.

    \begin{cor}
        Given a graph $G$, $Q\left(G\right)$
        is either edgeless, complete or prime.
    \end{cor}   
    
    One reason why
    quotient graphs $Q\left(G\right)$
    decided by maximal strong
    modules of a graph $G$ 
    are so useful, is
    that, while $Q\left(G\right)$
    is smaller that $G$, 
    it still contains a some
    useful information about
    $G$ itself. We express
    this fact through the following lemma.

    \begin{lemma} \label{prime lemma}
        Let $G$ and $Q\left(G\right)$
        be defined as above.
        Let $X$ be an
        induced subgraph
        of $G$. Then,
        if $X$ is prime, either
        it is contained into 
        a maximal strong module
        of $G$ or $X$ is 
        an induced subgraph of $Q\left(G\right)$.

        Conversely, if $X$ is any induced subraph of $Q\left(G\right)$,
        then $X$ is also a induced subgraph of $G$.
    \end{lemma}
    \begin{proof}
        For any maximal strong module $M$ of
        $G$, the set $M' \defeq V\left(X\right) \cap M$
        is a module of $X$. Therefore
        it has to be either a singleton or
        the entire $V\left(X\right)$.
        In the latter case, $X$ is a subgraph
        of $G\left[M\right]$.
        In the former,
        by the definition
        of modularity
        and by the fact
        that maximal strong
        modules partition
        $V\left(X\right)$,
        the maximal strong modules
        intersecting $X$ induce a subgraph
        in $Q\left(G\right)$ which is
        isomorphic to $X$.

        Conversely, if $X$ is 
        an induced subgraph
        in $Q\left(G\right)$,
        we define 
        $S$ to be the
        set containing exactly
        one representative of each
        module of $G$ which is
        also a vertex of $X$.
        Then, by the definition
        of $Q\left(G\right)$,
        the induced
        subgraph $G\left[S\right]$ 
        is isomorphic to $X$.
    \end{proof}

    This lead us directly to
    one of the main propositions 
    of \cite{main}.
    
    \begin{prop} \label{main prop}
        Let $* \in \left\{\text{I}, \text{LI}\right\}$. We have that
        $G \in \mathcal{F}_{*}$ and does not have a universal vertex
        if and only if all of the following holds.
        \begin{enumerate}
            \item The quotient graph $Q$ decided by maximal strong modules
                is not a clique; \label{1}
            \item $Q$ is in $\mathcal{F}_{*}$; \label{2}
            \item $G[M] \in \mathcal{F}_{*}$ for all modules $M$ which are simplicial in $Q$; \label{3}
            \item $G[M]$ is a clique for all modules $M$ which are non-simplicial in $Q$. \label{4}
        \end{enumerate}
    \end{prop}
    \begin{proof}
        We prove the forward direction by contraposition.
        
        Suppose that $Q$ is a clique. 
        If there is a module $M$ deciding $Q$
        which forms a clique in $G$, we have a universal vertex.
        Thus, take two modules $M$ and $M'$ deciding $Q$ and consider
        vertices $v_1, v_2 \in M$ and  $u_1, u_2 \in M'$ 
        with $v_1 \not \sim v_2$ and $u_1 \not \sim u_2$.
        This forms an induced $4$-hole.

        Suppose that a quotient graph $Q$ decided my maximal strong
        modules of $G$ is not in $\mathcal{F}_{*}$.
        Then, there is a graph $H \in \mathcal{F}_{*}$ induced in $Q$.
        Thus $H$ is an induced subgraph in $G$.

        Suppose that $G[M] \not \in \mathcal{F}_{*}$ for some module $M$ 
        which is simplicial in $Q$. Then $G[M]$ 
        contains an induced graph $H \in \mathcal{F}_{*}$. Thus, so does $G$.

        Suppose that $G\left[M\right]$ is not a clique for $M$ 
        a non-simplicial vertex of $Q$.
        Then, we get that $\left|M\right| \geq 2$ and that
        there are $M_1$ and $M_2$ adjacent to $M$ in $Q$
        such that $M_1$ and $M_2$ are not adjacent.
        Take two non-adjacent $u_1, u_2 \in M$ and
        any $v_1 \in M_1$ and $v_2 \in M_2$.
        We have that $u_1, v_1, u_2, v_2$ form
        an induced $4$-hole since $v_1$ and $v_2$ are
        not adjacent in $G$.
            
        \vspace{5pt}

        We prove the backward direction by contraposition.

        Suppose that $G \not \in \mathcal{F}_{*}$ with $G$ with no universal vertices.
        Thus, $Q$ is not a clique. This settles point \ref{1}.

        Since $G \not \in \mathcal{F}_{*}$, $G$ has an induced subgraph $H \in \mathcal{F}_{*}$.
        If $H$ is prime, i.e., $H$ is not a $4$-hole, 
        then either $H$ is an induced subgraph in $Q$ or in one of the
        modules deciding $Q$ (otherwise it would contradict the primality of $H$).
        If $H$ is not prime, i.e. a  $4$-hole,
        it either is completely
        contained into one module or 
        each vertex is contained into 
        a different module (thus
        settling point \ref{2} and \ref{3}),
        or it is contained in two or three modules.
        
        In this last case notice that at least one module
        contains exactly two non-adjacent vertices.
        So it is not a clique module.
        We also claim that at least two non-adjacent
        vertices of the $4$-hole lie in a non-simplicial vertex.
        Indeed, suppose that the vertices of the $4$-hole
        lie in exactly three modules, then two of these
        modules must contain
        exactly one vertex of the $4$-hole
        and must be non-adjacent
        while being both adjacent to the
        third one, which is thus non-simplicial. 
        Otherwise, suppose that all the vertices of the $4$-hole, 
        lie in exactly two modules $M_1$ and $M_2$.
        If $M_1$ is simplicial, then there must be a module $M_3$
        not adjacent to $M_1$, because otherwise, either Q is a
        complete graph or $\left\{M_1, M_2\right\}$ forms a module
        of $Q$, which is prime (since it is not a clique
        nor it is edgless, since $M_1$ and $M_2$ are
        adjacent),
        a contradiction. 
        Therefore $M_1$ and $M_3$ are not adjacent and
        $M_2$ is not simplicial.
        
        We just found a non-simplicial vertex in $Q$
        which contains two non-adjacent vertices, and thus
        it is not a clique module. This settles point \ref{4}.
    \end{proof}
 
    Proposition \ref{main prop} tells us that,
    if we ensure that
    $G$ does not have universal
    vertices, and
    once $Q\left(G\right)$ is
    compted by $\texttt{compute\_quotient}\left(G\right)$,
    in order to ensure that
    $G$ is interval,
    we only need to check that 
    $Q\left(G\right)$ is interval
    itself,
    non-simplicial modules are cliques and
    that $G\left[M\right]$ is
    an interval graph for all simplicial
    modules $M$.

    In order to
    achive the minimal 
    number of modifications
    while enforcing
    condition \ref{3}
    of Proposition \ref{main prop},
    we would like that,
    if $G\left[M\right]$ is not
    and interval graph, the
    graph obtained by substituting
    $G\left[M\right]$ by its
    largest induced interval subgraph
    is part of some minimum
    of modifications required to
    transform $G$ into an
    interval graph.
    This is indeed the case,
    provided that $Q\left(G\right)$
    is $4$-hole free.
    Before proving so,
    we fist make an observation.

    Maximal strong modules
    enjoy a remarkable ``modularity''
    property.
    Let $G$ be a graph such
    that $Q\left(G\right)$ is
    a prime graph. Let $M$ be
    a module of $G$ and let
    $H$ be a graph of
    vertex set $M'$.
    Consider replacing $G\left[M\right]$ 
    by $H$ in such a way 
    that $u \in V \left(G\right) \setminus M$
    contained in the maximal
    strong module $M'' \in V\left(Q\left(G\right)\right)$
    and $v \in M'$ are
    adjacent if and only if
    $M$ and $M''$ are adjacent
    in $V\left(Q\right)$.
    Then we have the following.
    
    \begin{lemma} \label{for thm}
        Consider the graph $G'$
        obtained after replacing
        $G\left[M\right]$ by $H$.
        Then, $M'$ is a maximal
        strong module of $G'$ 
        and $Q\left(G\right)$
        is isomorphic to
        $Q\left(G'\right)$.
    \end{lemma}
    \begin{proof}
    First, we notice that, by construction,
    $M'$ and all the modules of $G$
    except for $M$ are still modules of $G'$.
    We are now going to prove that such modules
    are stong and maximal in $G'$.
    By abuse of notation,
    we denote $V\left(Q\left(G\right)\right)$ 
    by $V\left(Q\right)$
    and $V\left(Q\left(G'\right)\right)$ 
    by $V\left(Q'\right)$.

    We first prove that they are strong by contradiction.
    Let $X$ be a module of $G'$ such that $M' \cap X \neq \emptyset$
    and $M' \not \subseteq X$ and $X \not \subseteq M$.
    Let $\left\{M_1, M_2, \ldots, M_{p}\right\}$ be the set of maximal strong modules
    in $V\left(Q\right)$ such 
    that $X \cap M_{i} \neq \emptyset$.
    If $X$ is a module of $G'$, then, since modules are either completely
    adjacent or completely non adjacent, we deduce that $\left\{M_1, M_2, \ldots, M_{p}\right\}
    \setminus \left\{M'\right\}$
    forms a module of $Q$. Since such module is non-trivial ($M$ is
    not in the module)
    and $Q$ is prime, we get a contradiction.
    Therefore $X$ is not a module of $G'$ and $M'$ is strong in $G'$.

    Let $N \in V\left(Q\right) \setminus \left\{M\right\}$.
    Using the strength of $M'$ in $G'$ and that of $N$ in $G$, 
    we deduce that $N$ is strong in $G'$.
    
    Now we prove maximality. Suppose that for some $N \in \left(V\left(Q\right)
        \setminus \left\{M\right\} \right) \cup \left\{M'\right\}$
    such module is not maximal in $G'$. By the strength of the modules
    in $V\left(Q\right)$, we deduce that there exists
    $\Gamma' \subsetneq \left(V\left(Q\right) \setminus \left\{M\right\}\right) 
    \cup \left\{M'\right\}$ such that $N \in \Gamma'$ and
    $\bigcup_{L \in \Gamma'} L \subsetneq V\left(G'\right)$ is a module of $G'$.
    Let $\Gamma = \left(\Gamma' \setminus \left\{M'\right\}\right) \cup \left\{M\right\}$ 
    if $M' \in \Gamma'$ or $\Gamma = \Gamma'$ otherwise.
    The above observation imply that $\Gamma$ is a non-trivial module of $Q$, which is a contradiction.
    Therefore all the modules in $\left(V\left(Q\right) \setminus \left\{M\right\}\right)
    \cup \left\{M'\right\}$ are strong and maximal
    in $G'$ and cover $V\left(G'\right)$.
    
    It's clear by construction that $Q'$ 
    is isomorphic to $Q$.
    \end{proof}
    
    This property is useful because
    it allows to consider maximal
    strong modules as
    ``black boxes''
    which can contain any subgraph. 
    
    We are now ready to prove an
    important result.
    
    \begin{thm} (\cite{main}) \label{replacing thm}
        Let $G$ be a graph of which
        every $4$-hole is contained
        is some maximal strong module,
        and let $G\left[U\right]$ be a
        maximum induced interval
        subgraph of $G$.
        For any maximal strong module
        $M$ of $G$ intersecting $U$,
        the set $M \cap U$ is a module
        of $G\left[U\right]$, 
        and replacing $G\left[M \cap U \right]$ 
        by any maximum induced 
        interval subgraph of $G\left[M\right]$ 
        in $G\left[U\right]$ 
        gives a maximum induced interval
        subgraph of $G$.
    \end{thm}
    \begin{proof}
        The first assertion is proved
        by Lemma \ref{for thm}.
        We focus on the second one.

        Let $U'$  be the set of 
        vertices that induces the new graph.
        Since $G\left[M \cap U\right]$ is
        interval, by the maximality of
        $M \cap U'$, we deuce that
        $\left|U'\right| \geq \left|U\right|$.

        We are left to prove that
        $G\left[U'\right]$ is interval.
        By way of contradiction,
        suppose that $G\left[U'\right]$ 
        is not interval.
        Thus, there exists a
        subgraph $X \in \mathcal{F}_{I}$
        in $G\left[U'\right]$.
        Since $G$ is $4$-hole free, 
        by Proposition \ref{main prop},
        at least one of $M \cap U'$ 
        and $N\left(M\right) \cap U'$ 
        induces a clique in $G$.
        Since $G$ is $4$-hole free,
        $X$ must be prime and,
        by Lemma \ref{prime lemma},
        $X$ contains exactly
        one vertex of $M$.
        Call such vertex $x$.
        By assumption, 
        there exists $x' \in M \cap U$.
        Let $X' \defeq X \setminus \left\{x\right\} \cup \left\{x'\right\}$.
        We thus have $X' \subseteq U$,
        but $G\left[X'\right]$ is 
        isomorphic to $G\left[X\right]$.
        Thus contradicting the fact
        that $G\left[U\right]$ is an
        interval graph.
    \end{proof}

    This discussions
    suggests to strip $G$ of
    its universal vertices and
    compute $Q\left(G\right)$.
    If $Q\left(G\right)$ is
    edgeless, then we
    solve
    each module individually.
    Or, if $Q\left(G\right)$ is 
    a clique, we have two cases. 
    Either
    find two
    non-clique modules of $Q\left(G\right)$,
    and thus we can easily find
    a $4$-hole in $G$ (just like
    in the proof
    of Proposition \ref{main prop})
    and branch on
    deleting one vertex of it.
    We can also find only one
    non-clique module,
    in which case we solve
    it individually.
    We cannot have only clique
    modules in this case, 
    otherwise we would have
    a universal vertex.
    Finally, if $Q\left(G\right)$
    is prime,
    Theorem \ref{replacing thm} 
    and \ref{main prop}
    tell us that if we
    transform $Q\left(G\right)$ into
    an interval graph (in particular,
    a $4$-hole free graph)
    a solution is given by
    solving individually
    the simplicial modules
    in $Q\left(G\right)$. 
    
    Therefore, from
    now on, we will be looking
    only at the quotient graph $Q\left(G\right)$,
    which we will denote by $Q$ 
    for the sake of brevity.
    Moreover, we will be referring
    to the vertices of $Q$ as the
    \emph{modules} of $Q$,
    not to be confused with the 
    \emph{vertices} of the
    original graph $G$.
    We will also keep
    looking for
    minimal modifications
    turning $Q$ in an
    interval graph (if it is not
    one already). Only,
    in this case, 
    the minimality
    of modifications will
    not consist in deleting
    the minimal number of modules
    of $Q$ but in deleting
    the modules of $Q$ so
    to minimize the total
    number of vertices
    contained in such
    modules.
    For any set $S$ of modules of $Q$ 
    we denote by $\#S$ the cardinality
    of their union (which is the
    sum of their cardinality,
    since they are disjoint).

    \section{Clique decompositions}

    We now introduce the notion of
    a clique decomposition of a graph $Q$.

    \begin{defn}
        Given a graph $Q$, we call
        a clique decomposition of $Q$ a 
        connected graph $\mathcal{K}$ whose vertex set
        is the set of (inclusion-wise)
        maximal cliques of $Q$ (called bags) and
        edge set such that for all
        $v \in Q$, the subgraph
        induced by the set of bags containing
        $v$ is connected.
    \end{defn}

    It turns out that interval
    graphs are characterized by
    their clique decomposition.
    
    \begin{prop} \label{clique-path}
        A graph $Q$ is an 
        interval graph if and only if
        it has a clique-path decompostion.
    \end{prop}
    \begin{proof}
        Suppose that a graph $Q$ 
        is interval.
        Let $\mathcal{I} = \left(I_{v}\right)_{v \in V\left(Q\right)}$ 
        be an interval model of it.
        We assume without loss
        of generality that
        all $I_{v}$ are closed
        intervals, all intersections
        $I_{v} \cap I_{v'}$ are
        not singletons and that
        such interval model is
        normalized, that is,
        $\inf \left\{a \mid \left[a, b\right] \in \mathcal{I} \right\} = 0$
        and $\sup \left\{b \mid \left[a, b\right] \in \mathcal{I}\right\} = 1$.

        For any $a \in \left[0, 1\right]$, 
        we define $V\left(a\right) = \left\{v \mid a \in I_{v}\right\}$.
        Now, we define inductively the
        family $\left(a_0, \ldots, a_{l}\right)$ 
        such that $a_i \in \left[0, 1\right]$ 
        for all $0 \leq i \leq l$.

        We define $a_0 \defeq 0$ and
        $a_{i+1} \defeq \sup \left\{a \in \left[a_{i}, 1\right] \mid 
        V\left(a_{i}\right) \subseteq V\left(a\right)\right\}$.
        
        Let $\left(a_0, a_1, \ldots, a_{l}\right)$ 
        be the family obtained at 
        the end of this construction.
        It is clear that $V\left(a_1\right), \ldots, V\left(a_{l}\right)$ 
        are the maximal cliques of $G$.
        Moreover, if $v \in V\left(a_{i}\right) \cap V\left(a_{k}\right)$,
        then $v \in V\left(a_{j}\right)$
        for all $i < j < k$.
        Thus yielding a clique-path 
        decomposition of $Q$.

        Conversely, let $M_0, \ldots, M_{l-1}$ 
        be a clique-path decompostion
        of a graph $Q$. 
        Let $M \defeq \bigcup_{i=0}^{l-1} M_{i}$.
        For all $v \in M$,
        let $\texttt{left}\left(v\right)$
        (resp. $\texttt{right}\left(v\right)$)
        be the smallest
        (resp. largest) index $i$ 
        such that $v \in M_{i}$.
        For $v \in V\left(Q\right)$,
        let $I_{v} \defeq \left[\texttt{left}\left(v\right) - \frac{1}{3}, 
        \texttt{right}\left(v\right) + \frac{1}{3}\right]$.
        The interval model $\mathcal{I} = \left(I_{v}\right)_{v \in V\left(Q\right)}$
        represents a graph with the
        given clique-path decomposition.
    \end{proof}

    Given Proposition \ref{clique-path},
    in order to transform
    a graph $Q$ into an
    interval graph, it suffices
    to to modify
    (always with the minimal
    number of vertex deletions)
    a clique decomposition of
    $Q$ into a clique-path decomposition.
    
    Moreover, as proved in \cite{inter-path},
    one can check in linear time whether
    a graph is interval and, in case of 
    a prositive answer, output
    a clique-path decomposition of it.
    We will denote such function by
    $\texttt{compute\_clique\_path}\left(\cdot\right)$.

    \section{Olive-ring decomposition}

    In this section we will give
    an outline of the results 
    and the ideas behind the 
    notion of the olive-ring
    decomposition used in the
    vertex deletion algorithm
    in \cite{main}.

    An olive-ring is a
    graph made of a main cycle
    with all other vertices
    attached to it having 
    degree one.

    At this point of 
    the algorithm,
    the author of \cite{main} focuses
    on finding an olive-ring
    decomposition of $Q$
    by first computing
    a hole
    decomposition of $Q - S\left(Q\right)$.
    The result is very technical
    but we would still like
    to give a glimpse into
    the technique used.
    The proofs of the
    statements in this
    section require
    some pretty
    involved constructions
    but they are not
    very hard to check
    by hand.
    For the detailed
    explanations we
    refer the reader to
    sections $4$ and $5$ 
    of \cite{main}.
    
    We start by using the
    algorithm $\texttt{find\_hole}\left(Q\right)$
    to find a hole in $Q$ if
    there is one.
    If none is found, then we
    can apply the following lemma.

    \begin{lemma} (\cite{main}) \label{large caw lemma}
        Let $W$ be a large caw
        of a prime graph $Q$, 
        then we can in find
        a subgraph of $Q$ 
        in $\mathcal{F}_{LI}$ 
        if the shallow terminal of $W$
        is non-simplicial in $Q$.
    \end{lemma}

    This lemma tells us 
    that, if the graph $Q$
    is chordal, then
    the graph $Q - S\left(Q\right)$ 
    is either interval,
    or we can return a
    subgraph of $Q$ in $\mathcal{F}_{LI}$.
    We can subsequently
    compute a clique-path 
    decomposition of $Q - S\left(Q\right)$ 
    with the linear-time
    subroutine $\texttt{compute\_clique\_path}\left(Q - S\left(Q\right)\right)$.
    and then join the two ends
    of the path without
    breaking any of the 
    conditions of
    the definition
    of a clique decomposition,
    thus getting a
    clique hole decomposition
    of $Q - S\left(Q\right)$.
   
    On the other end,
    if $Q$ is not
    chordal, we must use a
    more sophisticated tool.
    We first suppose that 
    the hole returned by
    the function $\texttt{find\_hole}\left(Q\right)$ 
    is long, otherwise,
    we just return the
    short hole.
    We have the following
    technical lemma which 
    allows us to give 
    a more precise description 
    of the structure
    of the graphs we will
    be considering from now on.
    
    \begin{lemma} (\cite{main}) \label{fund lemma}
        Let $H$ be a hole
        of a prime graph $Q$.
        We can, in linear time,
        find a subgraph of $Q$
        in $\mathcal{F}_{LI}$
        if there exists a vertex $v$ 
        satisfying one of the
        following three
        conditions:
        \begin{enumerate}
            \item the neighbours
                of $v$ in $H$ 
                are not consectuive;
            \item $v$ is adjacent
                to $\left|H\right| - 2$ 
                or more vertices in $H$;
            \item $v$ is non-simplicial and nonadjacent
               to $H$. \label{cond 3}
        \end{enumerate}
        Moreover, if there exists
        a subset $U \subseteq Q$, 
        such that $Q\left[U\right]$ 
        is connected and the
        neighbors of $U$ in
        $H$ do not induce
        a subpath, we can 
        return in linear
        time a subgraph of $Q$ 
        in $\mathcal{F}_{LI}$.
    \end{lemma}
    
    Therefore, if $Q$ is a
    locally interval graph,
    for any vertex $h$ of $H$, 
    the subgraph $Q - N\left[h\right]$ 
    must be chordal, otherwise
    $h$ and any hole of $Q - N\left[h\right]$
    will satisfy condition \ref{cond 3}
    of Lemma \ref{fund lemma}.
    Moreover, combining lemmas
    \ref{large caw lemma} 
    and \ref{fund lemma},
    we deduce that $Q - N\left[h\right] - S\left(Q\right)$ 
    is an interval graph.
    In order to exploit
    this observation,
    we require the more 
    sophisticated notion
    of an auxiliary graph.

    From now on, when we
    consider a hole $H$, 
    we will label its consecutive
    vertices as $h_{0}, h_{1}, \ldots, h_{\left|H\right|-1}$,
    where indices are understood modulo
    $\left|H\right|$. Note that to univocally 
    assign these labels, it suffices to
    specify two consecutive 
    vertices $h_0$ and $h_1$
    of $H$.
    
    By lemma \ref{fund lemma},
    we can assume that
    the vertices in $N\left[v\right] \cap H$
    induce a subpath for any vertex
    $v$ and any fixed hole $H$.
    Thus, once we are given
    a hole $H$, we can
    define for all vertices
    $v$ adjacent to it,
    the indices $\texttt{first}\left(v\right)$
    and $\texttt{last}\left(v\right)$
    such that 
    \begin{itemize}
        \item $-\left|H\right| < \texttt{first}\left(v\right)
            \leq 0 \leq \texttt{last}\left(v\right) < \left|H\right|
            \text{, if } v \sim h_{0}\text{; or}$ 
        \item $0 < \texttt{first}\left(v\right) 
            \leq \texttt{last}\left(v\right) < \left|H\right|
            \text{, otherwise.}$
    \end{itemize}
    
    Now, starting from 
    the hole returned
    by $\texttt{find\_hole}\left(Q\right)$, we
    want to find a potentially new hole
    on which we
    enforce some local
    properties on one of its 
    vertices (in fact, on $h_0$).
    We have the following.

    \begin{lemma} (\cite{main}) \label{hole lemma}
        We can, in linear time,
        find either a subgraph
        of $Q$ in $\mathcal{F}_{LI}$, 
        or a hole $H$ with
        a vertex $h_0$ such that
        for every vertex $v \in N\left(h_{-1}\right) \cap  N\left(h_{1}\right)$
        we have
        $N\left[v\right] \subseteq N\left[h_0\right]$.
    \end{lemma}
    
    From now on, 
    we assume that the hole $H$
    we are working with
    has the properties
    stated in Lemma \ref{hole lemma}.
    We observe that by
    Lemma \ref{fund lemma},
    we can assume that for
    all vertices 
    $v$ adjacent to $H$, 
    $N\left[v\right] \cap V\left(H\right)$
    induces a subpath.
    Also, by Lemma \ref{hole lemma},
    we can assume that 
    every common neighbor $v$ 
    of $h_1$ and $h_{-1}$ 
    is adjacent to $h_0$.
    Lemma \ref{hole lemma}
    also tells us
    that for any $v$ adjacent
    to $h_0$, $v \not\sim h_2$ 
    and $v \not\sim h_{-2}$,
    since $N\left[v\right] \subseteq N\left[h_0\right]$.
    We can also assume
    that such $v$ is adjacent to at
    least one of $h_1$ or $h_{-1}$.
    Otherwise, since $v \sim h_0$,
    $v \not\sim h_2$ and $v \not \sim h_{-2}$,
    if $u$ is adjacent to $H$,
    we call the second part of 
    Lemma \ref{fund lemma}
    with $U = \left\{v, u\right\}$;
    otherwise, we just
    return \emph{long claw}
    $\left\{u, v, h_0, h_1, h_{2}, h_{-2}\right\}$.
    Finally, for any $v \sim h_0$,
    such that $v \sim u$ with
    $u \not\sim h_0$,
    we can assume that
    $v$ is not 
    adjacent to all
    of $h_0$, $h_1$ and $h_{-1}$. 
    Otherwise, we
    would have $N\left[v\right] \subseteq N\left[h_0\right]$ 
    and thus $u \in N\left[h_0\right]$,
    which is a contradiction.
    All in all, we have
    that for all $v \sim h_0$,
    such that $v \sim u$,
    with $u \not\sim h_0$,
    we can assume that
    $v$ is adjacent to
    exactly one of $h_1$
    and $h_{-1}$ and so,
    that either $\texttt{first}\left(v\right)
    < \texttt{last}\left(v\right) = 0$,
    or $0 = \texttt{first}\left(v\right) < \texttt{last}\left(v\right)$.
    We are now ready to 
    give the definition of 
    the auxiliary graph $\mho\left(Q\right)$.

    \begin{defn} \label{aux graph}
        Given a graph $Q$, 
        and a hole $H$ of $Q$ 
        with the properties
        required by Lemma \ref{hole lemma}. Let
        $T \defeq N\left[h_0\right]$,
        let $L$ and $R$ be two distinct
        copies of $T$ and
        let $\overline{T} \defeq V\left(Q\right) \setminus T$.
        In order to distinguish
        two copies of the same
        vertex $v \in T$,
        we denote by $v^{l}$ 
        its copy in $L$ 
        and by $v^{r}$ its 
        copy in $R$.
        The vertex set $\mho\left(Q\right)$
        is the union $\overline{T} \cup L \cup R \cup \left\{w\right\}$,
        where $w$ is an additional fresh vertex.
        For every edge $uv \in E\left(Q\right)$,
        we add to the edge set
        of $\mho\left(Q\right)$
        \begin{itemize}
            \item an edge $uv$ if neither $u$ not $v$ is in $T$;
            \item two edges $u^{l}v^{l}$ and $u^{r}v^{r}$ 
                if both $u$ and $v$ are in $T$ ;
            \item an edge $uv^{l}$ if $u \in \overline{T}$,
                $v \in T$ and $0 = \texttt{first}\left(v\right)
                < \texttt{last}\left(v\right)$.
            \item an edge $uv^{r}$ and
                an edge $wv^{l}$ 
                if $u \in \overline{T}$,
            $v \in T$ and $\texttt{first}\left(v\right)
            < \texttt{last}\left(v\right) = 0$.
        \end{itemize}
    \end{defn}
    
    Notice that, by this definition, 
    we have $h_1^{l} \sim h_{2}$ and
    $h_{-2} \sim h_{-1}^{r}$.
    Also, the addition
    of the vertex $w$ is done so that 
    we can ``remember'' 
    which vertices in $R$ are
    connected to $\overline{T}$ 
    by looking at vertices in $L$.
    We add that the auxiliary graph of $Q$ is also
    cheap to build. 

    \begin{prop} (\cite{main})
        The number of vertices (resp. edges) of the graph
        $\mho\left(Q\right)$ are upper bounded 
        by $2 \left|G\right|$ (resp. $2 \|G\|$).
        Moreover, an adjacency matrix of $\mho\left(Q\right)$
        and $\mho\left(Q - S\left(Q\right)\right)$,
        can be built in linear time.
    \end{prop}

    The idea behind Definition \ref{aux graph}
    is that the auxilary graph either
    ``opens up'' all holes of $Q$,
    or we can find subgraphs of $Q$ in
    $\mathcal{F}_{LI}$ in linear time. 
    To make this more precise, we 
    give the following lemma.

    \begin{lemma} (\cite{main}) 
        If $\mho\left(Q\right)$ is not chordal,
        then we can find a subgraph of
        $Q$ in $\mathcal{F}_{LI}$.
    \end{lemma}
   
    We also have a version of
    Lemma \ref{large caw lemma}
    for $\mho\left(Q\right)$.

    \begin{lemma}
        If $\mho\left(Q - S\left(Q\right)\right)$
        is not an interval graph,
        then we can
        find a subgraph of 
        $Q$ in $\mathcal{F}_{LI}$.
    \end{lemma}

    Thus, we can assume
    that $\mho\left(Q - S\left(Q\right)\right)$
    is an interval graph
    which we can use to
    construct a hole
    decomposition of $Q - S\left(Q\right)$.
    Notice that, given
    a vertex $v$ and
    a set of vertices
    $S$, one
    can check in
    $\mathcal{O}\left(d\left(v\right)\right)$
    time whether $v$ 
    is adjacent to $S$ 
    or not. Thus, we can
    check in linear time
    whether any of 
    the modules of $Q - S\left(Q\right)$ 
    is adjacent to the set
    $L \cup R \cup \left(\overline{T} \cap V\left(H\right)\right)$.
    If any such module is found,
    then, it is surely not
    simplicial in $Q$,
    we can thus call
    Lemma \ref{fund lemma}
    on $H$ and $v$ as input
    and return a subgraph
    of $Q$ in $\mathcal{F}_{LI}$.
    We can thus assume 
    that all vertices of
    $Q - S\left(Q\right)$ 
    are adjacent to $H$ 
    and so, all modules
    in $\overline{T}$
    are adjacent to
    one among $h_1^{l}, h_2, \ldots h_{-1}^{r}$.
    Let $\mathcal{C}$ be the
    clique-path decomposition
    of $\mho\left(Q - S\left(Q\right)\right)$
    with consecutive bags
    numbered $K_0, K_1, \ldots, K_{\left|\mathcal{C}\right| - 1}$.
    By the definition of
    the auxiliary graph, 
    one can
    show that either the vertex $w$ 
    is simplicial in $\mho\left(Q - S\left(Q\right)\right)$ or
    we can return a
    subgraph of $Q$ in
    $\mathcal{F}_{LI}$
    in linear time \cite{main}.
    So, we can assume
    that $w$ is simplicial
    in $\mho\left(Q - S\left(Q\right)\right)$
    and thus
    its closed neighborhood
    $N\left[w\right]$, is
    a maximal clique of
    $\mho\left(Q - S\left(Q\right)\right)$.
    In this case we have
    that $N\left[w\right]$ 
    is an end bag of
    $\mathcal{C}$.
    We can argue this
    by showing that
    every neighbor 
    of $N\left[w\right]$ 
    is connected to $h_0^{l}$ 
    in $\mho\left(Q - S\left(Q\right)\right) - N\left[w\right]$.
    In fact, we have that
    every vertex in
    $L \setminus N\left[w\right]$,
    is adjacent to $h_0^{l}$,
    by definition.
    Also, every vertex of 
    $\overline{T}$ is conected
    to one of $h_1^{l}, h_2, \ldots, h_{-1}^{r}$.
    Since neither $h_0^{l}$ nor $h_0^{r}$ 
    are simplicial in $\mho\left(Q\right)$,
    we get that $h_0^{l}, h_1^{l}, h_2, \ldots,
    h_{-1}^{r}, h_0^{r}$ remains a path
    in $\mho\left(Q - S\left(Q\right)\right) - N\left[w\right]$.
    Thus $\mho\left(Q - S\left(Q\right)\right) - N\left[w\right]$
    is connected and $N\left[w\right]$
    is an end bag of $\mathcal{C}$.
    Without loss of generality,
    assume that $N\left[w\right] = K_0$.
    Notice that since $w$ is
    simplicial, the set
    $S = \left\{v^{r} \mid \text{ there exists
    $u$ with } v^{r} \sim u\right\}$ 
    is a clique. Moreover, 
    it is the minimal 
    clique separator of
    $\left\{w\right\} \cup L \cup \overline{T}$
    and $R \setminus S$.
    Therefore $h_0^{r}$ 
    only appears in bags
    whose index is larger
    than the largest index in
    of the bag containing $S$.
    Let $l$ be the largest
    index such that $h_0^{r} \not\in K_{l}$,
    One can show that the graph $\mathcal{K}$
    constructed by taking
    the subpath $K_1, \ldots, K_{l}$ 
    and joining its ends is
    a clique hole decomposition
    of $Q - S\left(Q\right)$.
    
    All in all, we have the following 
    theorem.

    \begin{thm} (\cite{main})
        Given a graph $G$, we
        can in linear time, either
        build its quotient graph
        $Q$ and a hole decomposition
        of $Q - S\left(Q\right)$,
        or output a subgraph
        in $\mathcal{F}_{LI}$.
        
        Moreover, if such hole
        decomposition is found,
        one can retrieve in linear
        time a normal Helly
        circular-arc
        interval model of $Q - S\left(Q\right)$
        as well as the shortest hole 
        of $Q$ (if there is one).
    
        Finally, if such a hole 
        decomposition is found,
        its consecutive bags are
        labeled as $K_0, K_1, \ldots, K_{\left|\mathcal{K}\right|-1}$,
        with $\left|\mathcal{K}\right| \leq \left|Q\right|$.
        \label{olive-ring}
    \end{thm}

    We remark right away
    that, when we compute
    the shortest hole of $Q$,
    if such hole is short,
    that is, it is either 
    a $4$ or a $5$-hole, 
    we branch on deleting one vertex of it.
    We can thus assume
    that the shortest
    hole of $Q$ has
    six or more vertices.
        
    The advantage of having a hole
    decomposition $\mathcal{K}$ of $Q$ is 
    that every hole of $Q$
    spans all the bags of $\mathcal{K}$.

    \begin{prop} \label{hole prop}
        Given a hole decomposition $\mathcal{K}$ of
        a graph $Q$,
        for all bags $K$ of $\mathcal{K}$,
        and every hole $H$ of $Q$,
        there exists a vertex $v$ 
        of $H$ contained in $K$.
    \end{prop}
    \begin{proof}
        By way of contradiction,
        suppose that there exists a 
        bag $K$ which does not 
        contain any vertex of a hole $H$.
        
        For all vertces $v$ of $H$, 
        let $K_{v}$ denote the
        set of bags containing $v$. 
        Let $K'$ be the closest bag
        to $K$ (say counterclockwise in $\mathcal{K}$)
        containing a vertex $v$ of $H$.
        Let $u, w$ be the two neighbors
        of $v$ in $H$.
        Since $u \not\sim w$, we have
        $K_{u} \cap K_{w} = \emptyset$.
        Thus, at least one of $K_{u}$ 
        and $K_{w}$ is contained
        in $K_{v}$. Suppose
        without loss of generality
        that $K_{u} \subseteq K_{v}$.
        Let $u' \neq v$ be the other
        neighbor of $u$
        (so $u' \neq w$ since $H$ has at 
        least four vertices). Since
        $K_{u} \cap K_{u'} \neq \emptyset$
        and $K_{u} \subseteq K_{v}$,
        we have $K_{u'} \cap K_{v} \neq \emptyset$
        and so $v \sim u'$, which
        gives us a contradiction.
    \end{proof}

    \begin{rmk} \label{hole rmk}
        Thus, it suffices to make the hole
        decomposition
        into a path to break all the holes
        in the original graph.
        However, in our case, we want
        to achieve the 
        minimal number of 
        deletions needed 
        to break such holes.      
    \end{rmk}

    We now introduce some notation.
    For any vertex $v$, let 
    $\texttt{left}\left(v\right)$
    and $\texttt{right}\left(v\right)$ 
    be defined as the indices such that
    $v \not \in K _{\texttt{left}\left(v\right)-1}$,
    $v \not \in K_{\texttt{right}\left(v\right)+1}$ 
    and $v \in K_{i}$ for all 
    $i \in \left[\texttt{left}\left(v\right), \texttt{right}\left(v\right)\right]$,
    where indices here are understood
    modulo $\left|\mathcal{K}\right|$.
    The existence of the
    above indices is
    assured by the fact that 
    the bags containing
    any vertex $v$ induce a proper subpath
    in $\mathcal{K}$.

    With this notation, we are
    ready to present another
    important result.

    \begin{lemma} (\cite{main}) \label{or_2}
        Given a hole decomposition
        of $Q - S\left(Q\right)$,
        we can, in linear time,
        construct a clique decomposition
        of $Q$ which is an olive-ring.
        Moreover, a module of $Q$ 
        does not appear in any
        bag of the main cycle
        if and only if it is
        a terminal of a caw
        of which all other modules are in
        the main cycle.
    \end{lemma}
    \begin{proof}
        Before presenting the proof, 
        we remark that for any $s \in S\left(Q\right)$ 
        the set $N\left[s\right]$ 
        is an inclusion-wise maximal
        clique. It is thus a bag
        of any clique decomposition of $Q$.

        Let $\mathcal{K}$ be the
        hole decomposition of $Q - S\left(Q\right)$.
        Traversing all bags in $\mathcal{K}$,
        we record $\texttt{left}\left(v\right)$ 
        and $\texttt{right}\left(v\right)$ 
        for all $v$ modules of $Q$.
        We also record $\left|K_{i} \cap K_{i+1}\right|$
        for all $i = 0, \ldots, \left|\mathcal{K}\right|-1$.
        We finally store, for every 
        module $s \in S\left(Q\right)$,
        the values $p \defeq \max_{v \in N\left(s\right)} \texttt{left}\left(v\right)$ 
        and $q \defeq \min_{v \in N\left(s\right)} \texttt{right}\left(v\right)$,
        so that $N\left(s\right) \subseteq K_{l} \cap K_{l+1}$ 
        for all $l \in \left[p, q-1\right]$.

        We do the following.
        \begin{itemize}
            \item If $p = q$ and $K_{p} = N\left(s\right)$,
                then we add $s$ to $K_{p}$;
            \item if there exists $l$ 
                such that $l \in \left[p, q-1\right]$ 
                and $\left|K_{l} \cap K_{l+1}\right| = \left|N\left(s\right)\right|$,
                then we insert $N\left[s\right]$
                as a new bag between $K_{l}$ 
                and $K_{l+1}$;
            \item otherwise, add $N\left[s\right]$ 
                as a pendant bag to $K_{p}$.
        \end{itemize}

        It is easy to check that this
        pocedure yileds an olive-ring
        decomposition of $Q$. 
        Now, let $P\left(Q\right)$ 
        be the set of modules that
        appear only in pendant bags
        (so, clearly, $P\left(Q\right) \subseteq S\left(Q\right))$.
        We notice
        in passing, that the addition of
        the simplicial modules in the main
        cycle keeps true the property
        of the existence of a normal
        Helly circular-arc model
        of subgraph induced
        by the modules appearing 
        in the main cycle 
        of the decomposition,
        which in this case
        is $Q - P\left(Q\right)$.
        
        We now take care of the 
        second assertion of the lemma.
        Suppose that a module of $Q$
        is a terminal of a caw of which
        all other modules are in the
        main cycle. If such module
        appeared itself of the main
        cycle then, since no
        caw has a hole, 
        the fact of having a normal Helly
        circular-arc model of $Q - P\left(Q\right)$,
        would imply the existence of 
        an interval model of a caw, which
        is impossible. 

        On the other hand, let 
        $v$ be a module in $P\left(Q\right)$.
        So, there are vertices $x \in K_{p-1} \setminus K_{p}$ 
        and $y \in K_{q+1} \setminus K_{q}$.
        Since $\left(K_{l} \cap K_{l+1}\right) \setminus N\left(s\right)$
        is non-empty, there exists an
        $x-y$ path entirely contained
        in the main cycle and avoiding
        neighbors of $s$. We can also
        find in linear time
        an $s-x$ path with only one intermediate
        vertex $x' \in N\left(s\right)$
        such that $\texttt{right}\left(x'\right)=p$
        and an $s-y$ path with only one intermediate
        vertex $y' \in N\left(s\right)$ 
        such that $\texttt{left}\left(y'\right)=q$.
        Such path witness the at $\left\{s, x, y\right\}$.

        The union of these paths contains a caw.
        Since all other vertices are from the main 
        cycle, this caw necessarily contains $s$ 
        (since otherwise, as previously remarked,
        we would get an interval model of a caw,
        which is impossible).
        Since $s$ is simplicial, 
        it is a terminal of this caw.
        Which is the desired result.
    \end{proof}
    
    We will call the algorithm
    that either returns an olive ring
    decomposition or a subgraph in $\mathcal{F}_{I}$,
    $\texttt{decompose}\left(\cdot\right)$.
    
    \begin{cor}
        Since caws do not have circular-arc models,
        the subgraph $Q - P\left(Q\right)$ is
        a normal Helly circular-arc graph
        which is also maximal in $Q$.
    \end{cor}
    
    As a last remark,
    it is easy to see that,
    given a hole decomposition $\mathcal{K}$ 
    for $Q - P\left(Q\right)$, a circular-arc model
    with circle length $\left|\mathcal{K}\right|$ 
    for $Q - P\left(Q\right)$ is given by
    
    \begin{gather*}
    A_{v} =
    \begin{cases}
    \left[\texttt{left}\left(v\right) - \frac{1}{3}, 
    \texttt{right}\left(v\right) + \frac{1}{3}\right], & \text{if } \texttt{left}\left(v\right) > 0, \\
    \left[\left|\mathcal{K}\right| - \frac{1}{3}, 
    \texttt{right}\left(v\right) + \frac{1}{3}\right], & \text{if } \texttt{left}\left(v\right) = 0.
    \end{cases}
    \end{gather*}
    
    Moreover, since holes (of $Q - P\left(Q\right)$) are of length
    at least $4$, such circular arc model
    is also normal and Helly.
        
    \section{From olive-ring to path decomposition}
    Now that we have an olive-ring
    decomposition of $Q$, we only need to
    strip off its pendant bags and break
    its main cycle in order to get an interval graph.
    However, we need to do so using
    the minimum possible number of vertex
    deletions. 
    
    In order to strip off the pendant
    bags, we need to introduce the notion
    of a minimal frame of a caw.

    \begin{defn}
        Given a frame $F$ of a large caw, we
        denote by $\left[F\right]$
        the set of bags $\texttt{right}\left(l\right),
        \ldots, \texttt{left}\left(r\right)$,
        (the inclusion-wise minimal
        set of consecutive bags whose union contains
        all non-terminal modules of $F$ ).
        We denote by $I\left(F\right) = \bigcup_{l \in \left[
        \texttt{right}\left(l\right) + 1, \texttt{left}\left(r\right) - 1\right]} 
        K_{l} \setminus N\left(Q\right)\left(s\right)$,
        the set of internal modules of $F$.
        $I\left(F\right)$ is the set of
        modules that can be used to construct
        the base of a caw with frame $F$.
        
        $F$ is said to be minimal
        if there is no other frame $F'$ 
        such that $\left[F'\right] \subseteq \left[F\right]$.
    \end{defn}
    
    Now consider, if there is one,
    a minimal frame $F$ in $Q$.
    We have the following technical lemma
    whose proof can be found in \cite{main}.

    \begin{lemma} (\cite{main}) \label{frame lemma}
        We can, in linear time,
        check whether or not
        $Q$ has a caw.
        If it is large,
        we can also return
        a minimal frame $F$ of it
        such that the following
        holds.

        Let $G\left[U^{*}\right]$ be
        a maximum induced interval
        subgraph of $G$
        such that $U^{*}$ 
        intersects every module
        of $F$.
        Let $S_{i} \defeq K_{i} \cap K_{i+1} \setminus N_{Q}\left(s\right)$ 
        where $\texttt{right}\left(v\right) \leq i < \texttt{left}\left(v\right)$
        and let $l$ be such
        that $\#S_{l}$ is minimum.
        Then, there exists a maximum
        induced interval subgraph
        $G\left[U\right]$ 
        disjoint from all modules in $S_{l}$.
        Moreover, $S_{l}$ can be
        found in linear time.
    \end{lemma}
    
    We will denote the
    function of Lemma \ref{frame lemma}
    returning a minimal frame
    $F$ by $\texttt{find\_minimal\_frame}\left(\cdot\right)$.

    With this lemma, we ensure that
    there always exists a minimum
    solution that deletes either 
    one module from $F$ or all
    the modules in $S_{l}$.
    
    Even though the proof
    of Lemma \ref{frame lemma}
    is very technical, the
    idea behind it is simple.
    When we want to dispose
    of a caw, we can either
    branch on deleting one 
    vertex from its frame
    or delete the smallest
    $l-r$ clique separator
    in the subgraph (of $Q$) induced by
    $\left\{l, r\right\} \cup I\left(F\right)$.

    Suppose that $S_{l}$ is
    deleted.
    Then, we
    also dispose of all
    the caws with frames $F'$ 
    such that $\left[F\right] \subseteq \left[F'\right]$.
    Moreover, let $S_{l}$ 
    be defined as above. It
    is easy to see that we can
    insert the pendant bag $v$
    belonging to $F$
    in between $K_{l}$ and
    $K_{l+1}$ without breaking
    any condition of the definition 
    of a clique decomposition.
    This also preserves the fact
    that the graph $Q - \left(P\left(Q\right) \setminus v\right)$
    is a normal Helly circular-arc graph.
    This procedure
    allows us to strip
    all the pendant bags
    of the olive-ring
    decomposition possibly 
    reinserting some of 
    them in the main cycle
    obtaining a clique-hole decomposition.
    For the sake of notation,
    we renumber the
    bags of the clique-hole decomposition
    so that consecutive
    bags have consecutive
    indices.

    The last thing to
    understand is how to break
    all holes of $Q - P\left(Q\right)$
    (if any are left)
    given a hole decomposition $\mathcal{K}$.
    We introduce the notion of a
    hole cover of a graph $G$. 

    \begin{defn}
        Given a graph $G$,
        a \emph{hole cover} of
        $G$ is a set of vertices $V_{-}$ 
        such that $G - V_{-}$ is a
        chordal graph.
    \end{defn}

    We have the following lemma.
    
    \begin{lemma}
        Given any clique-hole decomposition
        $\mathcal{K}$ of a normal Helly
        circular-arc graph $Q$, the hole
        covers of $Q$ are given by the
        the sets that contain the intersection
        of consecutive bags.
    \end{lemma}
    \begin{proof}
        On the one hand, take $M \defeq K_{l} \cap K_{l+1}$ 
        for $K_{l}$ and $K_{l+1}$ two consecutive
        bags of $\mathcal{K}$. 
        By Proposition \ref{hole prop},
        all intersections of consecutive
        cliques contain at least one vertex
        (actually, exactly two) of
        any hole.
        Thus $M$ is a hole cover of $Q$.

        On the other hand, take a hole separator
        $M$ of $Q$. Then it is a
        clique and it must contain
        the intersection of two consecutive bags.
        If not, we can find the minimum non-trivial
        cycle $C$ from any vertex in $K_0 \cap K_1$
        to itself that avoids $M$. Such non-trivial
        cycle exists because for any two consecutive
        bags of $\mathcal{K}$ there is a vertex $v$ 
        in their intersection which is not 
        contained in $M$.

        Since $Q$ is a normal Helly circular-arc
        graph, $C$ must have four or more vertices.
        Moreover, by the minimality of $C$, it must be chordless.
        Thus, $C$ is a hole.
        Which concludes the proof.
    \end{proof}
    
    From this lemma the following proposition 
    is deduced immediately.

    \begin{prop} (\cite{main}) \label{cycle prop}
        Let $\mathcal{K}$ be the hole decomposition of
        $Q$. For any
        inclusion-wise 
        minimal hole cover $V_{-}$ of $Q$
        there exists an index $l$ 
        such that $V_{-} = K_{l} \cap K_{l+1}$.
    \end{prop}

    It follows that,
    the minimal modification
    that breaks all holes of $Q$ 
    is the deletion of the
    vertices in $K_{l} \cap K_{l+1}$ 
    so that $\# \left(K_{l} \cap K_{l+1}\right)$
    is minimal.

    \section{The algorithm}
    
    We are now ready to
    present the pseudocode
    of the discussed procedure in
    Algorithm \ref{main algo}.
    We recall that, whenever
    we find a small caw in
    any of our subroutines,
    we implicitly branch
    on deleting one vertex of it.

    \begin{algorithm} \label{main algo}
        \caption{Interval vertex deletion}

        \SetKwFunction{IVD}{\texttt{interval\_vertex\_deletion}}
        \SetKwProg{func}{}{}{}

        \SetKw{strip}{strip}
        \SetKw{branch}{branch}
        \SetKw{return}{return}
        \SetKw{find}{find}
        \SetKw{delete}{delete}

        \KwIn{A graph $G$ and an
        integer $k$ representing
        the maximum number deletions.}
        \KwOut{A minimal set $V_{\min}$
        of vertices of $G$ such
        that $G - V_{\min}$ is interval,
        ``No'' otherwise.}
  
        \vspace{5pt}

        \func{\IVD{$G$, $k$}}{

            $V_{\min} \gets \emptyset$\; $k_{\min} \gets k$\; 

            \If{$k_{\min} < 0$}{
                \return ``No''\;
            }
            \If{$\texttt{find\_forbidden}\left(G\right)$ = ``No''}{
                \return $V_{\min}$ \;
            }
            
            \strip $G$ of its universal vertices\;
            
            $Q \gets \texttt{compute\_quotient}\left(G\right)$\;
            \If{$Q$ is edgeless}{
                \For{each module $M$ of $Q$}{
                $\tilde{V}_{\min} \gets \texttt{interval\_vertex\_deletion}\left(G\left[M\right], k\right)$\;
                \If{$\tilde{V}_{\min} = \text{``No''}$}{
                        \return ``No''\;
                    }
                    $V_{\min} \gets V_{\min} \cup \tilde{V}_{\min}$\; $k_{\min} \gets k_{\min} - \left|\tilde{V}_{\min}\right|$\;
                }
                \return $V_{\min}$ \;
            }
            \If{$Q$ is a complete graph}{
                \While{there exist two non-clique modules of $Q$}{
                    \find a $4$-hole and \branch on deleting one vertex form it\;
                    $k_{\min} \gets k_{\min} - 1$\;
                }
                \find the only non-clique module of $Q$, call it $M$\;
                \return $\texttt{interval\_vertex\_deletion}\left(G\left[M\right], k_{\min}\right)$\; 
            }
            $\mathcal{K} \gets \texttt{decompose}\left(Q\right)$\;
            \If{a short hole or a small caw is found}{
                \branch on deleting one vertex from it\;
                $k_{\min} \gets k_{\min} - 1$\;
            }
            \While{$Q$ contains a non-simplicial module}{
                \find a $4$-hole and branch on deleting one vertex from it\;
                $k_{\min} \gets k_{\min} - 1$\;
            }
            \For{each simplicial module $M$ of $Q$}{
                $\tilde{V}_{\min} \gets \texttt{interval\_vertex\_deletion}\left(G\left[M\right], k_{min}\right)$\; \label{replacing line}
                \If{$\tilde{V}_{\min} = \text{``No''}$}{
                    \return ``No''\;
                }
                $V_{\min} \gets V_{\min} \cup \tilde{V}_{\min}$\;
                $k_{\min} \gets k_{\min} - \left|\tilde{V}_{\min}\right|$\;
            }
            \While{$\mathcal{K}$ is not a hole}{
                $F \gets \texttt{find\_minimal\_frame}\left(Q, \mathcal{K}\right)$ \;
                \If{a small caw or a short hole is found}{
                    \branch on deleting one vertex from it\;
                    $k_{\min} \gets k_{\min} - 1$ \;
                }
                \branch on deleting either a module from $F$ or $S_{l}$ (as defined in Lemma \ref{frame lemma}) \label{most branch} 
            }
            \find the index $l$ such that $\#\left(K_{l} \cap K_{l+1}\right)$ is minimal (as defined in Proposition \ref{cycle prop})\;
            \delete all vertices in $K_{l} \cap K_{l+1}$ \;
            \return $V_{\min}$.
        } 
        
    \end{algorithm}
    
    We note in passing that the vailidity of
    line \ref{replacing line} in
    Algorithm \ref{main algo}
    is given by Theorem \ref{replacing thm}.
    
    The correctness of Algorithm \ref{main algo}
    was shown throughout this text.
    As for its complexity, each
    branching step has at most
    $8$ sub-instances (this
    maximum is attained only
    at line \ref{most branch}) each
    of which can be computed
    in linear time. Therefore,
    there are at most $8^{k}$ 
    total sub-instances and 
    the total complexity of the agorithm
    is $\mathcal{O} \left(8^{k} \cdot \left(\left|G\right| + \|G\|\right)\right)$.
    In the case in which $G$ is 
    a connected graph, this reduces to
    $\mathcal{O} \left(8^{k} \cdot \|G\|\right)$.

    We also can turn this
    algorithm into an approximation
    scheme in the following way. 
    Whenever we return a small caw
    or a short hole, we delete the whole
    subgraph. When we branch on deleting
    some entire modules at line \ref{most branch}
    of Algorithm \ref{main algo},
    we only delete exactly one
    vertex from each module of
    the frame $F$ and exactly
    one vertex from the
    union of the modules in $S_{l}$.
    At least one of these
    deletions is in some
    minimum modification and
    so the number of minimum
    modifications of the
    remaining graph decreases
    by at least one.
    Since we also have
    that small caws and short
    holes have at most $6$ ($ \leq 8$)
    vertices, we have
    a performance ratio of $8$.
    Moreover, the algorithm clearly stops
    after at most $\mathcal{O} \left(\left|G\right|\right)$ 
    deletions and thus
    such algorithm can beg
    implemented in $\mathcal{O} \left(\left|G\right| \cdot \left(\left|G\right| + \|G\|\right)\right)$
    time.

    We have the following theorem

    \begin{thm}
        Algorithm \ref{main algo} solves
        the minimum vertex deletion
        problem for interval graphs
        in $\mathcal{O} \left(8^{k} \cdot \left(\left|G\right| + \|G\|\right)\right)$ time.
        Moreover there exists
        an $8-\mathcal{O} \left(\left|G\right| \cdot \left(\left|G\right| + \|G\|\right)\right)$
        approximation scheme of the 
        same problem.
    \end{thm}

    \section{Concluding remarks}
    
    The strategy used by
    the algorithm presented in
    this text that of first
    computing an olive-ring
    decomposition or returning
    a small caw or a short hole;
    and then
    stripping the leaves
    of the olive-ring decomposition
    and breaking its main cycle.

    When first approaching
    such algorithm, our
    goal was to achieve
    an FPT algorithm and/or an appoximation 
    scheme similar
    to the one presented in \cite{main},
    but relative to the 
    class of chordal graphs
    (also known as triangulated graphs),
    which are more general than
    interval graphs as they
    can be characterized as
    intersection graphs of 
    a family of subtrees of a given tree,
    which is a more complex
    topological object than
    an interval (see \cite{book}).
    The idea of computing
    a clique decomposition
    of which
    reflects a
    normal and Helly circular-arc
    property of the underlying
    graph might be
    useful in the solution
    of this problem.
    However, the clique
    decomposition
    might be much more
    involved than an olive-ring.
    Moreover, the ways
    to compute such decomposition
    remains unclear. 
    Indeed, the algorithm
    presented in this
    text, relies heavily
    on the fact that,
    whenever we find a small
    caw, we branch on
    disposing of it
    and get a graph
    with more precise
    combinatorial properties.
    The fact that holes
    are the only minimal
    obstructions of
    chordal graphs
    make the main
    technical lemmas
    of \cite{main}
    useless. The details
    that need to be filled
    in order to get
    efficiently to 
    a useful clique
    decomposition thus need
    some more pondering.

    \printbibliography
\end{document}

